\documentclass[10pt,a4paper]{article}
\usepackage[english]{babel}
\usepackage[utf8]{inputenc}
\usepackage{url}
\usepackage{csquotes}
\usepackage{amsmath}
\usepackage{amssymb}
\usepackage{isabelle,isabellesym}

\usepackage{color}

\usepackage[top=3cm,bottom=4.5cm]{geometry}

\definecolor{keyword}{RGB}{0,153,102}
\definecolor{command}{RGB}{0,102,153}
\isabellestyle{tt}
\renewcommand{\isacommand}[1]{\textcolor{command}{\textbf{#1}}}
\renewcommand{\isakeyword}[1]{\textcolor{keyword}{\textbf{#1}}}

% this should be the last package used
\usepackage{pdfsetup}

% urls in roman style, theory text in isabelle-similar-similar type-writer
\urlstyle{rm}
\isabellestyle{tt}

\title{ \textbf{Coupled and Contra-Similarity} \\ \Large and How to Compute Them }
\author{ Benjamin Bisping \qquad Luisa Montanari%
  \footnote{Technische Universit\"at Berlin, Germany,
    \url{https://bbisping.de}, \texttt{\{benjamin.bisping,luisa.montanari\}@tu-berlin.de}.} }
%
\date{\today}

\begin{document}

\maketitle

\begin{abstract}
\noindent
This theory surveys a range of definitions of \emph{coupled similarity} and \emph{contrasimilarity},
and proves properties relevant for algorithms computing their simulation preorders and equivalences.

Coupled similarity and contrasimilarity are two weak forms of bisimilarity for systems with
internal behavior.
They have outstanding applications in contexts where internal choices must transparently be
distributed in time or space, for example, in process calculi encodings or in action refinements.

Our key contribution is to characterize the coupled simulation and contrasimulation preorders by \emph{reachability games}.
We also show how the preexisting definitions coincide and that they can be reformulated using \emph{coupled delay simulations}.
We moreover verify a polynomial-time coinductive fixed-point algorithm computing the coupled simulation preorder.
Through reduction proofs, we establish that deciding coupled similarity is at least as complex
as computing weak similarity; and that contrasimilarity checking is at least as hard as trace inclusion
checking.
\end{abstract}

\tableofcontents

\section{Introduction}

This theory contains the formalizations of two lines of publication:

\begin{itemize}
  \item \emph{Computing coupled similarity:} Benjamin Bisping and Uwe Nestmann's TACAS 2019 paper~%
    \cite{bn2019coupledsimTacas} and Bisping's master thesis~\cite{bisping2018coupledsim},
    which can be found on \url{https://coupledsim.bbisping.de/bisping_computingCoupledSimilarity_thesis.pdf},
    establish the first decision procedures for coupled similarity checking.
    This is done through a game-based approach.
    Also, the work introduces the idea that $\tau$-sinks can be used to reduce from weak simulation
    preorder to coupled simulation preorder.
  \item \emph{Game characterization of contrasimilarity:} Benjamin Bisping and Luisa Montanari's
    EXPRESS/SOS 2021 paper~\cite{bm2021contrasimilarity} and Montanari's bachelor thesis provide
    the first game characterization of contrasimilarity.
    This theory extends this work by also showing a reduction from weak trace preorder to
    contrasimulation preorder and by linking the game to a modal characterization of contrasimilarity.
\end{itemize}

% include generated text of all theories
\input{session}

\phantomsection
\addcontentsline{toc}{section}{References}
\bibliographystyle{splncs04}
\bibliography{root}

\end{document}
