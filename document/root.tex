\documentclass[10pt,a4paper]{article}
\usepackage[english]{babel}
\usepackage[utf8]{inputenc}
\usepackage{url}
\usepackage{csquotes}
\usepackage{amsmath}
\usepackage{amssymb}
\usepackage{isabelle,isabellesym}

\usepackage{color}

\usepackage[top=3cm,bottom=4.5cm]{geometry}

\definecolor{keyword}{RGB}{0,153,102}
\definecolor{command}{RGB}{0,102,153}
\isabellestyle{tt}
\renewcommand{\isacommand}[1]{\textcolor{command}{\textbf{#1}}}
\renewcommand{\isakeyword}[1]{\textcolor{keyword}{\textbf{#1}}}

% this should be the last package used
\usepackage{pdfsetup}

% urls in roman style, theory text in isabelle-similar-similar type-writer
\urlstyle{rm}
\isabellestyle{tt}

\title{ \textbf{Coupled and Contra-Similarity} \\ \Large and How to Compute Them }
\author{ Benjamin Bisping \qquad Luisa Montanari%
  \footnote{Technische Universit\"at Berlin, Germany,
    \url{https://bbisping.de}, \texttt{\{benjamin.bisping,luisa.montanari\}@tu-berlin.de}.} }
%
\date{\today}

\begin{document}

\maketitle

\begin{abstract}
\noindent
This theory surveys a range of definitions of \emph{coupled similarity} and \emph{contrasimilarity},
and proves properties relevant for algorithms computing their simulation relations.

Coupled similarity and contrasimilarity are two weak forms of bisimilarity for systems with
internal behavior.
They have outstanding applications in contexts where internal choices must transparently be
distributed in time or space, for example, in process calculi encodings or in action refinements.

We show how the preexisting definitions coincide and that they can be reformulated using
\emph{coupled delay simulations}. Our key contribution is to characterize the
coupled simulation and contrasimulation preorders by reachability games. We moreover verify a
polynomial-time coinductive fixed-point algorithm computing the coupled simulation preorder.
Through reduction proofs, we establish that deciding coupled similarity is at least as complex
as computing weak similarity; and checking contrasimilarity at least as complex as trace inclusion
checking.
\end{abstract}

\tableofcontents

\section{Introduction}

\cite{bm2021contrasimilarity}

This theory accompanies Benjamin Bisping and Uwe Nestmann's TACAS 2019 paper \cite{bn2019coupledsimTacas}
and Benjamin Bisping's master's thesis ``Computing Coupled Similarity'' \cite{bisping2018coupledsim},
which can be found on \url{https://coupledsim.bbisping.de/bisping_computingCoupledSimilarity_thesis.pdf}.



% include generated text of all theories
\input{session}

\phantomsection
\addcontentsline{toc}{section}{References}
\bibliographystyle{splncs04}
\bibliography{root}

\end{document}
